\chapter{Matteus}
\renewcommand{\chaplabel}{Matteus}
\jChapter{1}

\jOpskrif{Die geslagsregister van Jesus Christus}


\jverse{1}{}{Die geslagsregister van Jesus Christus, die Seun van Dawid, die seun van Abraham:}
\jverse{2}{}{Abraham was die vader van Isak, Isak van Jakob, en Jakob van Juda en sy broers:}
\jverse{3}{}{Juda was die vader van Peres en Serag by Tamar, Peres van Gesron, en Gesron van Ram;}
\jverse{4}{}{Ram was die vader van Amminadab, Amminadab van Nagson, en Nagson van Salmon;}
\jverse{5}{}{Salmon was die vader van Boas by Ragab, Boas van Obed by Rut, Obed van Isai;}
\jverse{6}{}{Isai was die vader van koning Dawid.  \newline Dawid was die vader van Salomo by die vrou van Urija;}
\jverse{7}{}{Salomo was die vader van Rehabeam, Rehabeam van Baia en Abia van Asa;}
\jverse{8}{}{Asa was die vader van Josafat van Joram, en Joram van Ussia;}
\jverse{9}{}{Ussia was die vader van Jotam, Jotam van Agas, en Agas van Hiskia;}
\jverse{10}{}{Hiskia was die vader van Manasse, Manasse van Amon, en Amon van Josia;}
\jverse{11}{}{Josia was die vader van Jojagin en sy broers.  Toe is die volk weggevoer na Babiloni\"e toe.}
\jverse{12}{}{En na die Babiloniese ballingskap was dit so:  Jojagin was die vader van Sealti\"el, en Sealti\"el van Serubbabel;}
\jverse{13}{}{Serubbabel was die vader van Abihud, Abihud van Eljakim, en Eljakim van Asor;}
\jverse{14}{}{Asor was die vader van Sadok, Sadok van Agim, en Agim van Elihud;}
\jverse{15}{}{Elihud was die vader van Eleasar, Eleasar van Mattan, en Mattan van Jakob;}
\jverse{16}{}{Jakob was die vader van Josef, die man van Maria uit wie gebore is Jesus wat Christus genoem word.}
\jverse{17}{}{Altesaam was daar dus veertien geslagte van Abraham af tot by Dawid, veertien van Dawid af tot by die Babiloniese ballingskap en veertien van die Babiloniese ballingskap af tot by Christus.}


\jOpskrif{Die geboorte van Jesus Christus}

\jverse{18}{}{Hier volg nou die geskiedenis van die geboorte van Jesus Christus. 
\newline
Toe sy moeder Maria aan Josef verloof was, nog voor hulle getroud was, het dit geblyk dat sy swanger is.  Die swangerskap het van die Heilige Gees gekom.}
\jverse{19}{}{Haar verloofde, Josef, wat aan die wet van Moses getrou was maar haar tog nie in die openbaar tot skande wou maak nie, het om voorgeneem om die verlowing stilweg te verbreek.}
\jverse{20}{}{Terwyl hy dit in gedagte gehad het, het daar 'n engel van die Here in 'n droom aan hom verskyn en ges\^e:  ``Josef seun van Dawid, moenie bang wees om met Maria te trou nie, want wat in haar verwek is, kom van die Heilige Gees.}
\jverse{21}{}{Sy sal 'n Seun in die w\^ereld bring, en jy moet Hom Jesus noem, want dit is Hy wat sy volk van hulle sondes swal verlos.''}
\jverse{22}{}{Dit het alles gebeur sodat die woord wat die Here deur sy profeet ges\^e het, vervul sou word:}
\jverse{23}{}{``Die maagd sal swanger word en `n seun in die w\^ereld bring, en hulle sal Hom Immanuel noem.''}
\jverse{24}{}{Toe Josef uit die slaap wakker word, het hy gemaak soos die engel van die Here hom beveel het en met haar getrou.}
\jverse{25}{}{Hy het egter nie met haar omgang gehad voordat sy haar Seun in die w\^ereld gebring het nie.  En Josef het Hom Jesus genoem.}

\jChapter{2}

\jOpskrif{Besoek uit die ooste}
\jverse{1}{}{Jesus is in Betlehem in Judea gebore tydens die regering van koning Herodes.  Na Jesus se geboorte het daar sterrekykers uit die ooste in Jerusalem aangekom }
\jverse{2}{}{en gevra:  ``Waar is Hy wat as koning van die Jode gebore is?  Ons het sy ster sien opkom en ons het gekom om aan Hom hulde te bewys.}
\jverse{3}{}{Toe koning Herodes hiervan hoor, was hy, en die hele Jerusalem saam met hom, hewig ontsteld.}
\jverse{4}{}{Hy het toe die priesterhoofde en die skrifgeleerdes van die volk bymekaar geroep en hulle uitgevra oor waar die Christus gebore sou word.}
\jverse{5}{}{Hulle het hom geantwoord:  ``In Betlehem in Judea, want so is dit deur 'n profeet geskrywe:}
\jverse{6}{}{En jy, Betlehem, gebied van Juda, jy is beslis nie die kleinste onder die leiers van Juda nie.  Uit jou sal 'n leier voortkom wat vir my volk Israel 'n herder sal wees.''}
\jverse{7}{}{Daarna het Herodes die sterrekykers in die geheim ontbied en noukeurig by hulle vasgestel wanneer die ster verskyn het.}
\jverse{8}{}{Hy het hulle na Betlehem toe gestuur met die woorde:  ``Gaan doen noukeurig ondersoek na die Kindjie, en as julle Hom kry, laat my weet, sodat ek ook aan Hom hulde kan gaan bewys.''}
\jverse{9}{}{Nadat hulle die koning aangehoor het, het hulle vertrek; en kyk, die ser wat hulle sien opkom het, het hulle gelei totdat dit gaan staan het bo die plek waar die Kindjie was.}
\jverse{10}{}{Toe hulle die ster sien, was hulle baie bly.}
\jverse{11}{}{Hulle het in die huis ingegaan en die Kindjie saam met Maria, sy moeder, gesien, en hulle het gekniel en aan Hom hulde bewys.  Daarna het hull hulle reissakke oopgemaak en vir Hom geskenke uitgehaal:  goud, wierook en mirre.}
\jverse{12}{}{En omdat God hulle in 'n droom gewaarsku het om nie na Herodes toe terug te gaan nie, het hulle met 'n ander pad na hulle land toe teruggegaan.}


\jOpskrif{Die vlug na Egipte toe}
\jverse{13}{}{Nadat hulle vertrek het, het daar 'n engel van die Here in 'n droom aan Josef verskyn en ges\^e:  ``Staan op, neem die Kindjie en sy moeder en vlug na Egipte toe en bly daar totdat ek jou s\^e om terug te kom, want Herodes is van plan om die Kindjie te soek en Hom dood te maak.''}
\jverse{14}{}{Josef het toe opgestaan en in die nag die Kindjie en sy moeder geneem en na Egipte toe vertrek}
\jverse{15}{}{ en daar gebly tot die dood van Herodes.  So is vervul wat die Here deur 'n profeet ges\^e het:  ``Uit Egipte het Ek my Seun geroep.''}


\jOpskrif{Die kindermoord}
\jverse{16}{}{Toe Herodes agterkom dat die sterrekykers hom in die steek gelaat het, het hy woedend geword.  Hy stuur toe soldate en laat in Betlehem en die hele omgewing al die seuntjies van twee jaar en jonger doodmaak, ooreenkomstig die tyd wat hy noukeurig by die sterrekykers uitgevra het.}
\jverse{17}{}{Toe is die woord vervul wat die Here deur die profeet Jeremia ges\^e het:  }
\jverse{18}{}{``'n Gekerm word op Rama gehoor, 'n gehuil en groot gejammer:  Ragel treur oor haar kinders en wil nie getroos word nie, omdat hulle nie meer daar is nie.''}


\jOpskrif{Die terugkoms uit Egipte}
\jverse{19}{}{Na die dood van Herodes verksyn daar 'n engel vand ie Here in 'n droom aan Josef en Egipte }
\jverse{20}{}{en s\^e:  ``Staan op, neem die Kindjie en sy moeder en gaan terug na die land van Israel toe, wannt die mense wat die Kindjie se lewe gesoek het, is dood.''}
\jverse{21}{}{Joesef het toe opgestaan, die Kindjie en sy moeder geneem en na die land van Israel toe teruggegaan.}
\jverse{22}{}{Toe hy egter verneem dat Argelaos in die plek van sy vader Herodes oor Judea regeer, was Josef bang om daarheen te gaan.  Nadat God hom in 'n droom 'n aanwysing gegee het, is hy na die landstreek Galilea toe.}
\jverse{23}{}{Daar aangekom, het hy gaan woon in 'n dorp met die naam Nasaret.  So is vervul wat deur die profete ges\^e is:  Hy sal Nasarener genoem word.}

\jChapter{3}

\jOpskrif{Johannes die Doper en sy prediking}
\jverse{1}{}{In daardie tyd het Johannes die Doper in die woestyn van Judea begin preek }
\jverse{2}{}{en ges\^e:  ``Bekeer julle, want die koninkryk van die hemel het naby gekom.'' }
\jverse{3}{}{Hy is die een wat die profeet Jesaja bedoel het toe hy ges\^e het:  ``Iemand roep in die woestyn:  Maak die pad vir die Here gereeed, maak die paaie vir Hom reguit.''}
\jverse{4}{}{Hierdie Johannes het klere van kameelphaar gedra met 'n leerband om sy heupe, en hy het van sprinkane en veldheuning gelewe.}
\jverse{5}{}{In di\'e dae het Jerusalem en die hele Judea en die hele omgewing van die Jordaan na hom toe gestroom.}
\jverse{6}{}{Hulle het hulle sondes bely en hulle deur hom in die Jordaanrivier laat doop. }
\jverse{7}{}{Toe hy merk dat baie van die Farise\"ers en die Sadduse\"ers kom om deur hom gedoop te word, s\^e hy vir hulle: ``Julle slange, wie het julle wysgemaak dat julle die dreigende toorn kan ontvlug? }
\jverse{8}{}{Dra liewer vrugte wat bewys dat julle bekeer is.}
\jverse{9}{}{Julle moet julle ni verbeel dat julle by julleself kan s\^e:  ``Ons het Abraham as voorvader'' nie.  Dit s\^e ek vir julle:  God kan uit hierdie klippe kinders vir Abraham verwek.}
\jverse{10}{}{Die byl l\^e klaar teen die wortel van die bome.  Elke boom wat nie goeie vrugte dra nie, word uitgekap en in die vuur gegooi.}
\jverse{11}{}{Ek doop julle wel met water omdat julle julle bekeer het, maar Hy wat n\'a my kom, is my meerdere, en ek is nie eers werd om sy skoene uit te trek nie.  Hy sal julle met die Hewilige Gees en met vuur doop.}
\jverse{12}{}{Hy het sy skop in sy hand en Hy sal sy dorsvloer deur en deur skoonmaak.  Sy koring sal HNy na die skuur toe bring, maar die kaf sal Hy met 'n onblusbare vuur verbrand.''}

\jOpskrif{Die doop van Jesus}
\jverse{13}{}{In daardie tyd het Jesus van Galilea af na Johannes toe by die Jordaan gekom om deur hom gedoop te word.}
\jverse{14}{}{Maar Johannes het dit probeer verhinder deur te s\^e ``Ek moet eintlik deur U gedoop word, en U kom na my toe?''}
\jverse{15}{}{Jesus het hom genatwoord:  ``Nogtans moet jy dit nou doen, want op hierdie manier moet ons aan die wil van God voldoen.''  Daarna het Johannes ingestem.}
\jverse{16}{}{Jesus is tgoe gedoop en het dadelik daarna uit die water gekom.  Meteens het die hemel bokant Hom oopgegaan, en Hy het die Gees van God soos 'n duif sien neerdaal op op Hom kom.}
\jverse{17}{}{Daar was ook 'n stem uit die hemel wat ges\^e het:  ``Dit is my geliefde Seun.  Oor Hom verheug Ek My.''}


\jChapter{4}

\jOpskrif{Nuut}
\jverse{1}{}{ }
\jverse{2}{}{ }
\jverse{3}{}{ }
\jverse{4}{}{ }
\jverse{5}{}{ }
\jverse{6}{}{ }
\jverse{7}{}{ }
\jverse{8}{}{ }
\jverse{9}{}{ }
\jverse{10}{}{ }
\jverse{11}{}{ }
\jverse{12}{}{ }



\jChapter{6}

\jOpskrif{Die sorge van die lewe}
\jverse{25}{}{``Daarom s\^e ek vir julle:  Moet julle nie bekommer oor julle lewe, oor wat julle moet eet of drink nie, of oor julle liggaam, oor wat julle moet aantrek nie.  Is die lewe nie belangriker as kos en die liggaam as klere nie?}
\jverse{26}{}{Kyk na die wilde vo\"els:  Hulle saai nie em hulle oes nie en hulle maak nie in skure bymekaar nie; julle hemelse Vader sorg vir hulle.  Is julle nie baie meer werd as hulle nie?}
\jverse{27}{}{Trouens, wie van julle kan deur hom te bekommer sy lewe met een enkele uur verleng?}
\jverse{28}{}{En wat bekommer julle julle oor klere?  Let op hoe groei die veldlelies:  Hulle swoeg nie en hulle maak nie klere nie.}
\jverse{29}{}{Maar Ek s\^e vir julle:  Selfs Salomo in al sy prag was nie geklee soos een van hulle nie.}
\jverse{30}{}{As God dan die gras van die veld, wat vandag nog daar is en môre verbrand word, so versier, hoeveel te meer sal Hy dan nie vir julle sorg nie, julle kleingelowiges?}
\jverse{31}{}{``Julle moet julle dus nie bekommer en vra: `Wat moet ons eet of wat moet ons drink of wat moet ons aantrek?' nie.}
\jverse{32}{}{Dit alles is dinge waaroor die ongelowiges begaan is.  Julle hemelse Vader weet tog dat julle dit alles nodig het.}
\jverse{33}{}{Nee, beywer julle allereeds vir die koningkryk van God en vir die wil van God, dan sal hy julle ook al hierdie dinge gee.}
\jverse{34}{}{``Moet julle dus nie oor môre bekommer nie, want môre bring sy eie bekommernis. Elke dag bring sy eie bekommernis.  Elke dag bring genoeg moeilikheid van sy eie.''}





