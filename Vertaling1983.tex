\documentclass[twoside,twocolumn,letterpaper]{book}
\usepackage[usenames,dvipsnames,svgnames,table]{xcolor}
\usepackage{lettrine}
\usepackage{fancyhdr}
\usepackage{fixltx2e}

\newcommand{\jChapter}[1]{\par\bigskip\lettrine{{\textcolor{red}{#1}}}{}\markboth{\chaplabel\ #1:1}{\chaplabel\ #1:1}\renewcommand{\jnumChapters}{#1}}

\newcommand{\jverseFormat}[1]{#1}
\newcommand{\jChapterNumFormat}[1]{\textcolor{cyan}{\textbf{#1}}}

\newcommand{\jverse}[3]{\noindent{\jChapterNumFormat{#1}\markboth{\chaplabel\ \jnumChapters :#1}{\chaplabel\ \jnumChapters :#1}} #2{\jverseFormat{#3}}\par\smallskip\renewcommand{\jnumVerses}{#1}}

\newcommand{\jBracketWord}[1]{\emph{#1}}

\newcommand{\jParaSymbol}[0]{{}}

\raggedright

\setlength{\columnseprule}{0.0pt}

\pagestyle{fancy}
\fancyhf{}
\fancyhead[RO]{\leftmark}
\fancyhead[LE]{\rightmark}
\renewcommand{\headrulewidth}{0pt}
\setlength{\headwidth}{\textwidth}
\addtolength{\headwidth}{\marginparsep}
\addtolength{\headwidth}{\marginparwidth}

\newcommand{\chaplabel}{}
\newcommand{\jnumChapters}{0}
\newcommand{\jnumVerses}{0}

\begin{document}

\frontmatter

\title{Die Bybel}
\date{}
\author{}
\setlength{\columnseprule}{0pt}
\maketitle

%\tableofcontents
\mainmatter

\part*{Ou Testament}
\setlength{\columnseprule}{0.0pt}
\renewcommand{\jnumChapters}{0}
\chapter{Genesis}
\renewcommand{\chaplabel}{Genesis}
\jChapter{1}
God se skeppingswerk
\jverse{1}{}{In the begin het God die hemel en die aarde geskep.}
\jverse{2}{}{Die aarde was heeltemal onbewoonbaar, dit was donker op die diep waters, maar die Gees van God het oor die waters gesweef.}
\jverse{3}{}{Toe het God ges\^e: ``Laat daar lig wees!''  En daar was lig.}
%\chapter{Genesis}
\renewcommand{\chaplabel}{Genesis}
\jChapter{1}

\jOpskrif{God se skeppingswerk}

\jverse{1}{}{In die begin het God die hemel en die aarde geskep.}
\jverse{2}{}{Die aarde was heeltemal onbewoonbaar, dit was donker op die diep waters, maar die Gees van God het oor die waters gesweef.}
\jverse{3}{}{Toe het God ges\^e: ``Laat daar lig wees!''  En daar was lig.}
\jverse{4}{}{God het gesien die lig is goed, en Hy het die lig en die donker van mekaar geskei.}
\jverse{5}{}{God het die lig toe ``dag'' genoem, en die donker het Hy ``nag'' genoem.  Dit het aand geword en dit het m\^ore geword.  Dit was die eerste dag.}
\jverse{6}{}{Toe het God ges\^e:  ``Laat daar \'n gewelf wees tussen die waters om die waters van mekaar te skei.''}
\jverse{7}{}{So het dit gebeur.  God het die gewelf gemaak en die waters onder die gewelf geskei van die waters bo die gewelf.  }
\jverse{8}{}{God het die gewelf ``hemel'' genoem.  Dit het aand geword en dit het môre geword.  Dit was die tweede dag.}
\jverse{9}{}{Toe het God ges\^e:  ``Laat die waters onder die hemel op een plek bymekaarkom sodat die droë grond sigbaar word.''  So het dit gebeur.}
\jverse{10}{}{God het die dro\"e grond ``aarde'' genoem, en die waters wat bymekaargekom het, het Hy ``see'' genoem.  En God het gesien dit is goed.}


\jChapter{34}

\jOpskrif{Dina en die mense van Sigem}


\jverse{1}{}{Dina, die dogter wat deur Lea vir Jakob in die w\^ereld gebring is, het op 'n keer gaan kuier by die dogters van die omgewing.}
\jverse{2}{}{Toe Sigem seun van die Hewiet Gamor, 'n volksleier, haar sien, het hy haar gegryp en met haar gemeenskap gehad. Hy het haar verkrag.}
\jverse{3}{}{Daarna het hy al hoe meer begin hou van Dina dogter van Jakob. Hy het die meisie lief gekry en haar hart probeer wen.}
\jverse{4}{}{Hy het vir sy pa, vir Gamor gesê: ``Kry vir my daardie meisie as vrou.''}
\jverse{5}{}{Jakob het verneem dat sy dogter Dina deur Sigem onteer is, maar omdat Jakob se seuns met die vee in die veld was, het hy niks gedoen voor hulle terug was nie.}
\jverse{6}{}{Toe Gamor, Sigem se pa, na Jakob toe gaan om met hom te praat oor Dina, }
\jverse{7}{}{kom Jakob se seuns juis van die veld af by die huis aan en toe hulle hoor wat gebeur het, was die manne ontsteld. Hulle was baie kwaad, want Sigem het 'n gruwelike ding aangevang toe hy met die dogter van Jakob, 'n Israelitiese meisie, gemeenskap gehad het. So iets doen 'n mens nie.}
\jverse{8}{}{Gamor het vir hulle gesê:  ``My seun Sigem hou baie van die dogter uit julle huis. Laat haar tog met hom trou.}
\jverse{9}{}{Kom ons word familie van mekaar. Dan kan julle dogters met ons trou en ons dogters met julle.}
\jverse{10}{}{Dan bly julle hier by ons. Ons land is tot julle beskikking. Bly net hier of trek rond en kry julle eie grond hier by ons.''}
\jverse{11}{}{Toe s\^e Sigem vir Dina se pa en haar broers:  ``As julle instem, sal ek betaal wat julle ook al van my vra.}
\jverse{12}{}{Julle kan maar die vergoeding wat ek moet gee, net so groot maak soos julle wil. Ek sal ook nog enige ander geskenk gee wat julle vra. Gee net asseblief die meisie vir my dat ek met haar kan trou.''}
\jverse{13}{}{Jakob se seuns het egter nie vir Sigem en sy pa Gamor 'n reguit antwoord gegee nie, want die broers se suster Dina was onteer.}
\jverse{14}{}{Hulle het toe ges\^e: ``Ons kan nie ons suster sommer so laat trou met 'n man wat nie besny is nie. Dit is vir ons 'n ongehoorde ding.}
\jverse{15}{}{Ons kan net instem as julle soos ons word: besny al julle mans.}
\jverse{16}{}{Dan kan ons ons dogters met julle laat trou en kan ons met julle dogters trou. Dan sal ons by julle bly en ons sal een volk word.}
\jverse{17}{}{Maar as julle nie na ons luister nie en julle nie laat besny nie, sal ons ons suster vat en wegtrek.''}
\jverse{18}{}{Gamor en sy seun Sigem het hiertoe ingestem.}
\jverse{19}{}{Die jongman Sigem wou nie die saak laat sloer nie, want hy was lief vir Jakob se dogter. Omdat Gamor en Sigem die meeste gesag onder hulle mense gehad het, }
\jverse{20}{}{het hulle na die stadspoort toe gegaan en daar vir die mans van die stad gesê: }
\jverse{21}{}{``Hierdie vreemde mense het met goeie bedoelings na ons toe gekom. Hulle wil hier kom bly en met hulle kuddes in ons land rond trek: die land is groot en l\^e oop voor hulle. Dan kan ons met hulle dogters trou en hulle met ons s'n.}
\jverse{22}{}{Hierdie mense sal egter net by ons kom bly en saam met ons een volk word as al ons mansmense besny word soos dit by hulle is.}
\jverse{23}{}{Hulle kuddes, al hulle vee, hulle besittings, word ons eiendom as ons instem, en dan sal hulle by ons bly.''}
\jverse{24}{}{Al die mans van die stad het na Gamor en sy seun Sigem geluister, en hulle almal is besny.}
\jverse{25}{}{Drie dae later toe hulle baie pyn gehad het, het Jakob se twee seuns Simeon en Levi, die broers van Dina, elkeen sy swaard gevat en sonder om agterdog te wek, die stad ingegaan en al die mans doodgemaak, }
\jverse{26}{}{ook vir Gamor en sy seun Sigem. Hulle het vir Dina uit Sigem se huis uit weggevat en uit die stad uit gegaan.}
\jverse{27}{}{Jakob se seuns het toe op die slagoffers toegesak en die stad geplunder omdat hulle suster daar onteer is: }
\jverse{28}{}{kleinvee, beeste, donkies, alles in die stad en alles in die veld het hulle vir hulle gevat. }
\jverse{29}{}{Ja, hulle het alles geplunder wat daar in die stad was, hulle het alles wat in die huise was, buitgemaak; ook die kinders en vrouens is as gevangenes saamgevat.}
\jverse{30}{}{Toe sê Jakob vir Simeon en Levi: ``Julle het my in die moeilikheid gebring, want julle het my naam sleg gemaak by die mense van hierdie land, die Kana\:aniete en die Feresiete. Ek het maar 'n klein klompie mans by my. As die mense van die land almal saamstaan teen my, sal hulle my en my mense verslaan en ons om die lewe bring.''}
\jverse{31}{}{Toe antwoord hulle: ``Hy moes nie ons suster soos 'n hoer behandel het nie!''}



\part*{Nuwe Testament}
\setlength{\columnseprule}{0.0pt}
\renewcommand{\jnumChapters}{0}
\chapter{Matteus}
\renewcommand{\chaplabel}{Matteus}
\jChapter{1}

Die geslagsregister van Jesus Christus

\jverse{1}{}{Die geslagsregister van Jesus Christus, die Seun van Dawid, die seun van Abraham:}
\jverse{2}{}{Abraham was die vader van Isak, Isak van Jakob, en Jakob van Juda en sy broers:}
\jverse{3}{}{Juda was die vader van Peres en Serag by Tamar, Peres van Gesron, en Gesron van Ram;}
\jverse{4}{}{Ram was die vader van Amminadab, Amminadab van Nagson, en Nagson van Salmon;}
\jverse{5}{}{Salmon was die vader van Boas by Ragab, Boas van Obed by Rut, Obed van Isai;}
\jverse{6}{}{Isai was die vader van koning Dawid.  \newline Dawid was die vader van Salomo by die vrou van Urija;}
\jverse{7}{}{Salomo was die vader van Rehabeam, Rehabeam van Baia en Abia van Asa;}
\jverse{8}{}{Asa was die vader van Josafat van Joram, en Joram van Ussia;}
\jverse{9}{}{Ussia was die vader van Jotam, Jotam van Agas, en Agas van Hiskia;}
\jverse{10}{}{Hiskia was die vader van Manasse, Manasse van Amon, en Amon van Josia;}
\jverse{11}{}{Josia was die vader van Jojagin en sy broers.  Toe is die volk weggevoer na Babiloni\"e toe.}
\jverse{12}{}{En na die Babiloniese ballingskap was dit so:  Jojagin was die vader van Sealti\"el, en Sealti\"el van Serubbabel;}
\jverse{13}{}{Serubbabel was die vader van Abihud, Abihud van Eljakim, en Eljakim van Asor;}
\jverse{14}{}{Asor was die vader van Sadok, Sadok van Agim, en Agim van Elihud;}
\jverse{15}{}{Elihud was die vader van Eleasar, Eleasar van Mattan, en Mattan van Jakob;}
\jverse{16}{}{Jakob was die vader van Josef, die man van Maria uit wie gebore is Jesus wat Christus genoem word.}
\jverse{17}{}{Altesaam was daar dus veertien geslagte van Abraham af tot by Dawid, veertien van Dawid af tot by die Babiloniese ballingskap en veertien van die Babiloniese ballingskap af tot by Christus.}

\end{document}