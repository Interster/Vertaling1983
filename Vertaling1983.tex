\documentclass[twoside,twocolumn,letterpaper]{book}
\usepackage[usenames,dvipsnames,svgnames,table]{xcolor}
\usepackage{lettrine}
\usepackage{fancyhdr}
\usepackage{fixltx2e}

\newcommand{\jChapter}[1]{\par\bigskip\lettrine{{\textcolor{red}{#1}}}{}\markboth{\chaplabel\ #1:1}{\chaplabel\ #1:1}\renewcommand{\jnumChapters}{#1}}

\newcommand{\jverseFormat}[1]{#1}
\newcommand{\jChapterNumFormat}[1]{\textcolor{cyan}{\textbf{#1}}}

\newcommand{\jverse}[3]{\noindent{\jChapterNumFormat{#1}\markboth{\chaplabel\ \jnumChapters :#1}{\chaplabel\ \jnumChapters :#1}} #2{\jverseFormat{#3}}\par\smallskip\renewcommand{\jnumVerses}{#1}}

\newcommand{\jBracketWord}[1]{\emph{#1}}

\newcommand{\jParaSymbol}[0]{{}}

\newcommand{\jOpskrif}[1]{\begin{center}\textit{#1}\end{center}}




\raggedright

\setlength{\columnseprule}{0.0pt}

\pagestyle{fancy}
\fancyhf{}
\fancyhead[RO]{\leftmark}
\fancyhead[LE]{\rightmark}
\renewcommand{\headrulewidth}{0pt}
\setlength{\headwidth}{\textwidth}
\addtolength{\headwidth}{\marginparsep}
\addtolength{\headwidth}{\marginparwidth}

\newcommand{\chaplabel}{}
\newcommand{\jnumChapters}{0}
\newcommand{\jnumVerses}{0}

\begin{document}

\frontmatter

\title{Die Bybel}
\date{14 Februarie 2012}
\author{soos oorgetik deur Ni\"el Agenbag}
\setlength{\columnseprule}{0pt}
\maketitle

%\tableofcontents
\mainmatter

\part*{Ou Testament}
\setlength{\columnseprule}{0.0pt}
\renewcommand{\jnumChapters}{0}


\chapter{Genesis}
\renewcommand{\chaplabel}{Genesis}
\jChapter{1}

\jOpskrif{God se skeppingswerk}

\jverse{1}{}{In die begin het God die hemel en die aarde geskep.}
\jverse{2}{}{Die aarde was heeltemal onbewoonbaar, dit was donker op die diep waters, maar die Gees van God het oor die waters gesweef.}
\jverse{3}{}{Toe het God ges\^e: ``Laat daar lig wees!''  En daar was lig.}
\jverse{4}{}{God het gesien die lig is goed, en Hy het die lig en die donker van mekaar geskei.}
\jverse{5}{}{God het die lig toe ``dag'' genoem, en die donker het Hy ``nag'' genoem.  Dit het aand geword en dit het m\^ore geword.  Dit was die eerste dag.}
\jverse{6}{}{Toe het God ges\^e:  ``Laat daar \'n gewelf wees tussen die waters om die waters van mekaar te skei.''}
\jverse{7}{}{So het dit gebeur.  God het die gewelf gemaak en die waters onder die gewelf geskei van die waters bo die gewelf.  }
\jverse{8}{}{God het die gewelf ``hemel'' genoem.  Dit het aand geword en dit het môre geword.  Dit was die tweede dag.}
\jverse{9}{}{Toe het God ges\^e:  ``Laat die waters onder die hemel op een plek bymekaarkom sodat die droë grond sigbaar word.''  So het dit gebeur.}
\jverse{10}{}{God het die dro\"e grond ``aarde'' genoem, en die waters wat bymekaargekom het, het Hy ``see'' genoem.  En God het gesien dit is goed.}


\jChapter{34}

\jOpskrif{Dina en die mense van Sigem}


\jverse{1}{}{Dina, die dogter wat deur Lea vir Jakob in die w\^ereld gebring is, het op 'n keer gaan kuier by die dogters van die omgewing.}
\jverse{2}{}{Toe Sigem seun van die Hewiet Gamor, 'n volksleier, haar sien, het hy haar gegryp en met haar gemeenskap gehad. Hy het haar verkrag.}
\jverse{3}{}{Daarna het hy al hoe meer begin hou van Dina dogter van Jakob. Hy het die meisie lief gekry en haar hart probeer wen.}
\jverse{4}{}{Hy het vir sy pa, vir Gamor gesê: ``Kry vir my daardie meisie as vrou.''}
\jverse{5}{}{Jakob het verneem dat sy dogter Dina deur Sigem onteer is, maar omdat Jakob se seuns met die vee in die veld was, het hy niks gedoen voor hulle terug was nie.}
\jverse{6}{}{Toe Gamor, Sigem se pa, na Jakob toe gaan om met hom te praat oor Dina, }
\jverse{7}{}{kom Jakob se seuns juis van die veld af by die huis aan en toe hulle hoor wat gebeur het, was die manne ontsteld. Hulle was baie kwaad, want Sigem het 'n gruwelike ding aangevang toe hy met die dogter van Jakob, 'n Israelitiese meisie, gemeenskap gehad het. So iets doen 'n mens nie.}
\jverse{8}{}{Gamor het vir hulle gesê:  ``My seun Sigem hou baie van die dogter uit julle huis. Laat haar tog met hom trou.}
\jverse{9}{}{Kom ons word familie van mekaar. Dan kan julle dogters met ons trou en ons dogters met julle.}
\jverse{10}{}{Dan bly julle hier by ons. Ons land is tot julle beskikking. Bly net hier of trek rond en kry julle eie grond hier by ons.''}
\jverse{11}{}{Toe s\^e Sigem vir Dina se pa en haar broers:  ``As julle instem, sal ek betaal wat julle ook al van my vra.}
\jverse{12}{}{Julle kan maar die vergoeding wat ek moet gee, net so groot maak soos julle wil. Ek sal ook nog enige ander geskenk gee wat julle vra. Gee net asseblief die meisie vir my dat ek met haar kan trou.''}
\jverse{13}{}{Jakob se seuns het egter nie vir Sigem en sy pa Gamor 'n reguit antwoord gegee nie, want die broers se suster Dina was onteer.}
\jverse{14}{}{Hulle het toe ges\^e: ``Ons kan nie ons suster sommer so laat trou met 'n man wat nie besny is nie. Dit is vir ons 'n ongehoorde ding.}
\jverse{15}{}{Ons kan net instem as julle soos ons word: besny al julle mans.}
\jverse{16}{}{Dan kan ons ons dogters met julle laat trou en kan ons met julle dogters trou. Dan sal ons by julle bly en ons sal een volk word.}
\jverse{17}{}{Maar as julle nie na ons luister nie en julle nie laat besny nie, sal ons ons suster vat en wegtrek.''}
\jverse{18}{}{Gamor en sy seun Sigem het hiertoe ingestem.}
\jverse{19}{}{Die jongman Sigem wou nie die saak laat sloer nie, want hy was lief vir Jakob se dogter. Omdat Gamor en Sigem die meeste gesag onder hulle mense gehad het, }
\jverse{20}{}{het hulle na die stadspoort toe gegaan en daar vir die mans van die stad gesê: }
\jverse{21}{}{``Hierdie vreemde mense het met goeie bedoelings na ons toe gekom. Hulle wil hier kom bly en met hulle kuddes in ons land rond trek: die land is groot en l\^e oop voor hulle. Dan kan ons met hulle dogters trou en hulle met ons s'n.}
\jverse{22}{}{Hierdie mense sal egter net by ons kom bly en saam met ons een volk word as al ons mansmense besny word soos dit by hulle is.}
\jverse{23}{}{Hulle kuddes, al hulle vee, hulle besittings, word ons eiendom as ons instem, en dan sal hulle by ons bly.''}
\jverse{24}{}{Al die mans van die stad het na Gamor en sy seun Sigem geluister, en hulle almal is besny.}
\jverse{25}{}{Drie dae later toe hulle baie pyn gehad het, het Jakob se twee seuns Simeon en Levi, die broers van Dina, elkeen sy swaard gevat en sonder om agterdog te wek, die stad ingegaan en al die mans doodgemaak, }
\jverse{26}{}{ook vir Gamor en sy seun Sigem. Hulle het vir Dina uit Sigem se huis uit weggevat en uit die stad uit gegaan.}
\jverse{27}{}{Jakob se seuns het toe op die slagoffers toegesak en die stad geplunder omdat hulle suster daar onteer is: }
\jverse{28}{}{kleinvee, beeste, donkies, alles in die stad en alles in die veld het hulle vir hulle gevat. }
\jverse{29}{}{Ja, hulle het alles geplunder wat daar in die stad was, hulle het alles wat in die huise was, buitgemaak; ook die kinders en vrouens is as gevangenes saamgevat.}
\jverse{30}{}{Toe sê Jakob vir Simeon en Levi: ``Julle het my in die moeilikheid gebring, want julle het my naam sleg gemaak by die mense van hierdie land, die Kana\:aniete en die Feresiete. Ek het maar 'n klein klompie mans by my. As die mense van die land almal saamstaan teen my, sal hulle my en my mense verslaan en ons om die lewe bring.''}
\jverse{31}{}{Toe antwoord hulle: ``Hy moes nie ons suster soos 'n hoer behandel het nie!''}



\chapter{Eksodus}
\renewcommand{\chaplabel}{Eksodus}
\jChapter{1}

\jOpskrif{Vul}

\jverse{1}{}{Vul.}


\chapter{Levitikus}
\renewcommand{\chaplabel}{|Levitikus}
\jChapter{1}

\jOpskrif{Vul}

\jverse{1}{}{Vul.}


\chapter{Numeri}
\renewcommand{\chaplabel}{Numeri}
\jChapter{1}

\jOpskrif{Vul}

\jverse{1}{}{Vul.}


\chapter{Deuteronomium}
\renewcommand{\chaplabel}{Deuteronomium}
\jChapter{1}

\jOpskrif{Vul}

\jverse{1}{}{Vul.}


\chapter{Josua}
\renewcommand{\chaplabel}{Josua}
\jChapter{1}

\jOpskrif{Vul}

\jverse{1}{}{Vul.}


\chapter{Rigters}
\renewcommand{\chaplabel}{Rigters}
\jChapter{1}

\jOpskrif{Vul}

\jverse{1}{}{Vul.}


\chapter{Rut}
\renewcommand{\chaplabel}{Rut}
\jChapter{1}

\jOpskrif{Vul}

\jverse{1}{}{Vul.}


\chapter{1 Samuel}
\renewcommand{\chaplabel}{1 Samuel}
\jChapter{1}

\jOpskrif{Vul}

\jverse{1}{}{Vul.}


\chapter{2 Samuel}
\renewcommand{\chaplabel}{2 Samuel}
\jChapter{1}

\jOpskrif{Vul}

\jverse{1}{}{Vul.}


\chapter{1 Konings}
\renewcommand{\chaplabel}{1 Konings}
\jChapter{1}

\jOpskrif{Vul}

\jverse{1}{}{Vul.}


\chapter{2 Konings}
\renewcommand{\chaplabel}{2 Konings}
\jChapter{1}

\jOpskrif{Vul}

\jverse{1}{}{Vul.}


\chapter{1 Kronieke}
\renewcommand{\chaplabel}{1 Kronieke}
\jChapter{1}

\jOpskrif{Vul}

\jverse{1}{}{Vul.}


\chapter{2 Kronieke}
\renewcommand{\chaplabel}{2 Kronieke}
\jChapter{1}

\jOpskrif{Vul}

\jverse{1}{}{Vul.}


\chapter{Esra}
\renewcommand{\chaplabel}{Esra}
\jChapter{1}

\jOpskrif{Vul}

\jverse{1}{}{Vul.}


\chapter{Nehemia}
\renewcommand{\chaplabel}{Nehemia}
\jChapter{1}

\jOpskrif{Vul}

\jverse{1}{}{Vul.}


\chapter{Ester}
\renewcommand{\chaplabel}{Ester}
\jChapter{1}

\jOpskrif{Vul}

\jverse{1}{}{Vul.}


\chapter{Job}
\renewcommand{\chaplabel}{Job}
\jChapter{1}

\jOpskrif{Vul}

\jverse{1}{}{Vul.}


\chapter{Psalms}
\renewcommand{\chaplabel}{Psalms}

\jChapter{23}

\jOpskrif{Die Here is my herder}


\jverse{1}{}{'n Psalm van Dawid.  Die Here is my herder, ek kom niks kort nie.}
\jverse{2}{}{Hy laat my rus in groen weivelde.  Hy bring my by waters waar daar vrede is.}
\jverse{3}{}{Hy gee my nuwe krag.  Hy lei my op die regte paaie tot die eer van sy Naam.}
\jverse{4}{}{Selfs al gaan ek deur donker dieptes, sal ek nie bang wees nie, want U is by my.  In u hande is ek veilig.}
\jverse{5}{}{U laat my by 'n feesmaal aansit, terwyl my teëstanders moet toekyk.  U ontvang my soos 'n eregas, ek word oorlaai met hartlikheid.}
\jverse{6}{}{U goedheid en liefde sal my lewe lank by my bly en ek sal tuis wees in die huis van die Here tot in lengte van dae.}




\jChapter{118}

\jOpskrif{Die Here doen kragtige dade}


\jverse{1}{}{Loof die Here, want Hy is goed, aan sy liefde is daar geen einde nie!}
\jverse{2}{}{Israel moet getuig:  ``Aan sy liefde is daar geen einde nie."}
\jverse{3}{}{Die huis van A\:aron moet getuig:  ``Aan sy liefde is daar geen einde nie."}
\jverse{4}{}{Di\'e wat die Here dien, moet getuig:  ``Aan sy liefde is daar geen einde nie."}
\jverse{5}{}{In my nood het ek na die Here geroep, die Here het my gebed verhoor en my bevry.}
\jverse{6}{}{Die Here is by my, ek ken geen vrees nie; wat kan 'n mens aan my doen?}
\jverse{7}{}{Die Here is by my, Hy help my: ek sal neerkyk op di\'e wat my haat.}
\jverse{8}{}{Om by die Here te skuil, is beter as om op mense te vertrou.}
\jverse{9}{}{Om by die Here te skuil, is beter as om op ho\"e mense te vertrou.}


\jChapter{146}

\jOpskrif{Prys die Here!}


\jverse{1}{}{Prys die Here!  Ek wil die Here prys.}
\jverse{2}{}{Ek wil die Here prys so lank ek lewe, ek wil die lof van my God besing so lank ek daar is.}
\jverse{3}{}{Moenie op magtiges vertrou nie, nie op 'n mens nie;  hy kan jou nie red nie.}
\jverse{4}{}{Sy asem verlaat hom, hy word weer grond en sy planne is daarmee heen.}
\jverse{5}{}{Dit gaan goed met die mens wat sy hulp van die God van Jakob ontvang, die mens wie se hoop gevestig is op die Here, sy God.}
\jverse{6}{}{Die Here het die hemel en die aarde gemaak, die see en alles daarin.  Hy bly vir altyd getrou,}



SPREUKE VAN SALOMO

PREDIKER

HOOGLIED VAN SALOMO

JESAJA

JEREMIA

KLAAGLIEDERE VAN JEREMIA

ESEGIëL

DANIëL

HOSEA

JOëL

AMOS

OBADJA

JONA

MIGA

NAHUM

HABAKUK

SEFANJA

HAGGAI

SAGARIA

MALEAGI





\part*{Nuwe Testament}
\setlength{\columnseprule}{0.0pt}
\renewcommand{\jnumChapters}{0}

\chapter{Matteus}
\renewcommand{\chaplabel}{Matteus}
\jChapter{1}

\jOpskrif{Die geslagsregister van Jesus Christus}


\jverse{1}{}{Die geslagsregister van Jesus Christus, die Seun van Dawid, die seun van Abraham:}
\jverse{2}{}{Abraham was die vader van Isak, Isak van Jakob, en Jakob van Juda en sy broers:}
\jverse{3}{}{Juda was die vader van Peres en Serag by Tamar, Peres van Gesron, en Gesron van Ram;}
\jverse{4}{}{Ram was die vader van Amminadab, Amminadab van Nagson, en Nagson van Salmon;}
\jverse{5}{}{Salmon was die vader van Boas by Ragab, Boas van Obed by Rut, Obed van Isai;}
\jverse{6}{}{Isai was die vader van koning Dawid.  \newline Dawid was die vader van Salomo by die vrou van Urija;}
\jverse{7}{}{Salomo was die vader van Rehabeam, Rehabeam van Baia en Abia van Asa;}
\jverse{8}{}{Asa was die vader van Josafat van Joram, en Joram van Ussia;}
\jverse{9}{}{Ussia was die vader van Jotam, Jotam van Agas, en Agas van Hiskia;}
\jverse{10}{}{Hiskia was die vader van Manasse, Manasse van Amon, en Amon van Josia;}
\jverse{11}{}{Josia was die vader van Jojagin en sy broers.  Toe is die volk weggevoer na Babiloni\"e toe.}
\jverse{12}{}{En na die Babiloniese ballingskap was dit so:  Jojagin was die vader van Sealti\"el, en Sealti\"el van Serubbabel;}
\jverse{13}{}{Serubbabel was die vader van Abihud, Abihud van Eljakim, en Eljakim van Asor;}
\jverse{14}{}{Asor was die vader van Sadok, Sadok van Agim, en Agim van Elihud;}
\jverse{15}{}{Elihud was die vader van Eleasar, Eleasar van Mattan, en Mattan van Jakob;}
\jverse{16}{}{Jakob was die vader van Josef, die man van Maria uit wie gebore is Jesus wat Christus genoem word.}
\jverse{17}{}{Altesaam was daar dus veertien geslagte van Abraham af tot by Dawid, veertien van Dawid af tot by die Babiloniese ballingskap en veertien van die Babiloniese ballingskap af tot by Christus.}


\jOpskrif{Die geboorte van Jesus Christus}

\jverse{18}{}{Hier volg nou die geskiedenis van die geboorte van Jesus Christus. 
\newline
Toe sy moeder Maria aan Josef verloof was, nog voor hulle getroud was, het dit geblyk dat sy swanger is.  Die swangerskap het van die Heilige Gees gekom.}
\jverse{19}{}{Haar verloofde, Josef, wat aan die wet van Moses getrou was maar haar tog nie in die openbaar tot skande wou maak nie, het om voorgeneem om die verlowing stilweg te verbreek.}
\jverse{20}{}{Terwyl hy dit in gedagte gehad het, het daar 'n engel van die Here in 'n droom aan hom verskyn en ges\^e:  ``Josef seun van Dawid, moenie bang wees om met Maria te trou nie, want wat in haar verwek is, kom van die Heilige Gees.}
\jverse{21}{}{Sy sal 'n Seun in die w\^ereld bring, en jy moet Hom Jesus noem, want dit is Hy wat sy volk van hulle sondes swal verlos.''}
\jverse{22}{}{Dit het alles gebeur sodat die woord wat die Here deur sy profeet ges\^e het, vervul sou word:}
\jverse{23}{}{``Die maagd sal swanger word en `n seun in die w\^ereld bring, en hulle sal Hom Immanuel noem.''}
\jverse{24}{}{Toe Josef uit die slaap wakker word, het hy gemaak soos die engel van die Here hom beveel het en met haar getrou.}
\jverse{25}{}{Hy het egter nie met haar omgang gehad voordat sy haar Seun in die w\^ereld gebring het nie.  En Josef het Hom Jesus genoem.}

\jChapter{2}

\jOpskrif{Besoek uit die ooste}
\jverse{1}{}{Jesus is in Betlehem in Judea gebore tydens die regering van koning Herodes.  Na Jesus se geboorte het daar sterrekykers uit die ooste in Jerusalem aangekom }
\jverse{2}{}{en gevra:  ``Waar is Hy wat as koning van die Jode gebore is?  Ons het sy ster sien opkom en ons het gekom om aan Hom hulde te bewys.}
\jverse{3}{}{Toe koning Herodes hiervan hoor, was hy, en die hele Jerusalem saam met hom, hewig ontsteld.}
\jverse{4}{}{Hy het toe die priesterhoofde en die skrifgeleerdes van die volk bymekaar geroep en hulle uitgevra oor waar die Christus gebore sou word.}
\jverse{5}{}{Hulle het hom geantwoord:  ``In Betlehem in Judea, want so is dit deur 'n profeet geskrywe:}
\jverse{6}{}{En jy, Betlehem, gebied van Juda, jy is beslis nie die kleinste onder die leiers van Juda nie.  Uit jou sal 'n leier voortkom wat vir my volk Israel 'n herder sal wees.''}
\jverse{7}{}{Daarna het Herodes die sterrekykers in die geheim ontbied en noukeurig by hulle vasgestel wanneer die ster verskyn het.}
\jverse{8}{}{Hy het hulle na Betlehem toe gestuur met die woorde:  ``Gaan doen noukeurig ondersoek na die Kindjie, en as julle Hom kry, laat my weet, sodat ek ook aan Hom hulde kan gaan bewys.''}
\jverse{9}{}{Nadat hulle die koning aangehoor het, het hulle vertrek; en kyk, die ser wat hulle sien opkom het, het hulle gelei totdat dit gaan staan het bo die plek waar die Kindjie was.}
\jverse{10}{}{Toe hulle die ster sien, was hulle baie bly.}
\jverse{11}{}{Hulle het in die huis ingegaan en die Kindjie saam met Maria, sy moeder, gesien, en hulle het gekniel en aan Hom hulde bewys.  Daarna het hull hulle reissakke oopgemaak en vir Hom geskenke uitgehaal:  goud, wierook en mirre.}
\jverse{12}{}{En omdat God hulle in 'n droom gewaarsku het om nie na Herodes toe terug te gaan nie, het hulle met 'n ander pad na hulle land toe teruggegaan.}


\jOpskrif{Die vlug na Egipte toe}
\jverse{13}{}{Nadat hulle vertrek het, het daar 'n engel van die Here in 'n droom aan Josef verskyn en ges\^e:  ``Staan op, neem die Kindjie en sy moeder en vlug na Egipte toe en bly daar totdat ek jou s\^e om terug te kom, want Herodes is van plan om die Kindjie te soek en Hom dood te maak.''}
\jverse{14}{}{Josef het toe opgestaan en in die nag die Kindjie en sy moeder geneem en na Egipte toe vertrek}
\jverse{15}{}{ en daar gebly tot die dood van Herodes.  So is vervul wat die Here deur 'n profeet ges\^e het:  ``Uit Egipte het Ek my Seun geroep.''}


\jOpskrif{Die kindermoord}
\jverse{16}{}{Toe Herodes agterkom dat die sterrekykers hom in die steek gelaat het, het hy woedend geword.  Hy stuur toe soldate en laat in Betlehem en die hele omgewing al die seuntjies van twee jaar en jonger doodmaak, ooreenkomstig die tyd wat hy noukeurig by die sterrekykers uitgevra het.}
\jverse{17}{}{Toe is die woord vervul wat die Here deur die profeet Jeremia ges\^e het:  }
\jverse{18}{}{``'n Gekerm word op Rama gehoor, 'n gehuil en groot gejammer:  Ragel treur oor haar kinders en wil nie getroos word nie, omdat hulle nie meer daar is nie.''}


\jOpskrif{Die terugkoms uit Egipte}
\jverse{19}{}{Na die dood van Herodes verksyn daar 'n engel vand ie Here in 'n droom aan Josef en Egipte }
\jverse{20}{}{en s\^e:  ``Staan op, neem die Kindjie en sy moeder en gaan terug na die land van Israel toe, wannt die mense wat die Kindjie se lewe gesoek het, is dood.''}
\jverse{21}{}{Joesef het toe opgestaan, die Kindjie en sy moeder geneem en na die land van Israel toe teruggegaan.}
\jverse{22}{}{Toe hy egter verneem dat Argelaos in die plek van sy vader Herodes oor Judea regeer, was Josef bang om daarheen te gaan.  Nadat God hom in 'n droom 'n aanwysing gegee het, is hy na die landstreek Galilea toe.}
\jverse{23}{}{Daar aangekom, het hy gaan woon in 'n dorp met die naam Nasaret.  So is vervul wat deur die profete ges\^e is:  Hy sal Nasarener genoem word.}

\jChapter{3}

\jOpskrif{Johannes die Doper en sy prediking}
\jverse{1}{}{In daardie tyd het Johannes die Doper in die woestyn van Judea begin preek }
\jverse{2}{}{en ges\^e:  ``Bekeer julle, want die koninkryk van die hemel het naby gekom.'' }
\jverse{3}{}{Hy is die een wat die profeet Jesaja bedoel het toe hy ges\^e het:  ``Iemand roep in die woestyn:  Maak die pad vir die Here gereeed, maak die paaie vir Hom reguit.''}
\jverse{4}{}{Hierdie Johannes het klere van kameelphaar gedra met 'n leerband om sy heupe, en hy het van sprinkane en veldheuning gelewe.}
\jverse{5}{}{In di\'e dae het Jerusalem en die hele Judea en die hele omgewing van die Jordaan na hom toe gestroom.}
\jverse{6}{}{Hulle het hulle sondes bely en hulle deur hom in die Jordaanrivier laat doop. }
\jverse{7}{}{Toe hy merk dat baie van die Farise\"ers en die Sadduse\"ers kom om deur hom gedoop te word, s\^e hy vir hulle: ``Julle slange, wie het julle wysgemaak dat julle die dreigende toorn kan ontvlug? }
\jverse{8}{}{Dra liewer vrugte wat bewys dat julle bekeer is.}
\jverse{9}{}{Julle moet julle ni verbeel dat julle by julleself kan s\^e:  ``Ons het Abraham as voorvader'' nie.  Dit s\^e ek vir julle:  God kan uit hierdie klippe kinders vir Abraham verwek.}
\jverse{10}{}{Die byl l\^e klaar teen die wortel van die bome.  Elke boom wat nie goeie vrugte dra nie, word uitgekap en in die vuur gegooi.}
\jverse{11}{}{Ek doop julle wel met water omdat julle julle bekeer het, maar Hy wat n\'a my kom, is my meerdere, en ek is nie eers werd om sy skoene uit te trek nie.  Hy sal julle met die Hewilige Gees en met vuur doop.}
\jverse{12}{}{Hy het sy skop in sy hand en Hy sal sy dorsvloer deur en deur skoonmaak.  Sy koring sal HNy na die skuur toe bring, maar die kaf sal Hy met 'n onblusbare vuur verbrand.''}

\jOpskrif{Die doop van Jesus}
\jverse{13}{}{In daardie tyd het Jesus van Galilea af na Johannes toe by die Jordaan gekom om deur hom gedoop te word.}
\jverse{14}{}{Maar Johannes het dit probeer verhinder deur te s\^e ``Ek moet eintlik deur U gedoop word, en U kom na my toe?''}
\jverse{15}{}{Jesus het hom genatwoord:  ``Nogtans moet jy dit nou doen, want op hierdie manier moet ons aan die wil van God voldoen.''  Daarna het Johannes ingestem.}
\jverse{16}{}{Jesus is tgoe gedoop en het dadelik daarna uit die water gekom.  Meteens het die hemel bokant Hom oopgegaan, en Hy het die Gees van God soos 'n duif sien neerdaal op op Hom kom.}
\jverse{17}{}{Daar was ook 'n stem uit die hemel wat ges\^e het:  ``Dit is my geliefde Seun.  Oor Hom verheug Ek My.''}


\jChapter{4}

\jOpskrif{Nuut}
\jverse{1}{}{ }
\jverse{2}{}{ }
\jverse{3}{}{ }
\jverse{4}{}{ }
\jverse{5}{}{ }
\jverse{6}{}{ }
\jverse{7}{}{ }
\jverse{8}{}{ }
\jverse{9}{}{ }
\jverse{10}{}{ }
\jverse{11}{}{ }
\jverse{12}{}{ }

MARKUS

LUKAS

JOHANNES

DIE HANDELINGE VAN DIE APOSTELS







\chapter{Romeine}
\renewcommand{\chaplabel}{Romeine}
\jChapter{1}


ROM 1

1.  Van Paulus, 'n dienaar van Christus 
Jesus, geroep om apostel te wees, af-
gesonder vir die evangelie van God.
2.  Hierdie evangelie het Hy vooruit 
deur sy profete in die Heilige Skrif aan-
gekondig, 3.  end dit handel oor sy Seun.  As
mens is Hy uit die geslag van Dawid ge-
bore;  4.  op grond van sy opstanding uit die 
dood is Hy deur die Heilige Gees aange-
wys as die Seun van God wat met mag beklee is,
Jesus Christus on Here.  5.  Deur 
Hom het ek die genade ontvang om apos-
tel te wees onder al die heidennasies, om
tot eer van sy Naam mense tot geloof en
so tot gehoorsaamheid te bring.  6.  Julle is
hierby inbegrepe;  ook julle is geroep om 
aan Jesus Crhistus te behoort.
7.  Aan almal in Rome vir wie God lief-
het en wat Hy geroep het om aan Hom te 
behoort.
Genade en vrede vir julle van God ons 
Vader en die Here Jesus Christus!

Paulus se dank, verlange en verpligting

8.  Allereers dank ek my God deur Jesus
Christus oor julle almal, want in die hele
wêreld praat die mense oor julle geloof.
9.  God, wat ek met hart en siel dien deur
die evangelie van sy Seun te verkondig,
is my getuie dat ek gedurig in my gebede
aan julle dink.  10.  Elke keer as ek bid, vra
ek God of Hy dit nie uiteindelik vir my 
moontlik wil maak om julle te bosoek nie.
11.  Ek sien baie daarna uit om julle te be-
soek, want ek wil 'n geestelike gawe aan 
julle oordra om julle te versterk.  12.  Ek
bedoel dat, as ek by julle is, ons deur
mekaar se geloof bemoedig kan word, ek
deur julle s'n en julle deur myne.
13.  Broers, ek wil hê julle moet weet dat,
hoewel ek tot nog toe verhinder is, ek my 
dikwels voorgeneem het om julle te be-
soek om ook by julle, net soos by die 
ander heidennasies, mense vir Christus
te wen.  14.  Ek staan onder 'n verpligting
teenoor Grieke en nie-Grieke, teenoor
ontwikkeldes en ongeleerdes, 15.  en van-
daar my begeerte om ook aan julle in 
Rome die evangelie te verkondig.
16.  Ek skaam my nie oor die evangelie
nie, want dit 'n krag van God tot red-
ding van elkeen wat glo, in die eerste plek
die Jood, maar ook die nie-Jood.  17.  In die 
evangelie kom juis tot openbaring dat
God mense van hulle sonde vryspreek
enkel en alleen omdat hulle glo.  Dit is
soos daar geskrywe staan:  "Elkeen wat
deur God vrygespreek is omdat hy glo, 
sal lewe."

Die sonde en die straf

18.  God openbaar vanuit die hemel sy
toorn oor al die goddeloosheid en onge-
regtigheid van die mense wat die waar-
heid deur hulle ongeregtigheid probeer
onderdruk.  19.  Wat 'n mens van God kan 
weet, was immers binne hulle bereik,
want God het dit binne hulle bereik ge-
bring.  20.  Van die skepping van die wêreld
af kan 'n mens uit die werke van God 
duidelik aflei dat sy krag ewigdurend is
en dat Hy waarlik God is, hoewel dit 
dinge is wat 'n mens nie met die oog kan
sien nie.  Vir hierdie mense is daar dus
geen verontskuldiging nie, 21.  omdat, al 
weet hulle van God, hulle Hom nie as 
God eer en dank nie.  Met hulle redena-
sies bereik hulle niks nie, en deur hulle
gebrek aan insig bly hulle in die duister.
22.  Hulle gee voor dat hulle verstandig is,
maar hulle is dwaas.  23.  In die plek van die
heerlikheid van die onverganklike God 
stel hulle beelde wat lyk soos 'n vergank-
like mense of soos voëls of diere of slange.
24.  Daarom gee God hulle aan die drange
van hulle hart oor en aan sedelike onrein-
heid, sodat hulle hulle liggame onder 
mekaar onteer.  25.  Dit is hulle wat die 
waarheid van God verruil vir die leuen.
Hulle dien en vereer die skepsel in plaas
van die Skepper, aan wie die lof toekom
vir ewig.  Amen.  26.  Daarom gee God hulle
oor aan skandelike drifte.  Hulle vroue
verander die natuurlike omgang in 'n 
teen-natuurlike omgang.  27.  Net so laat 
vaar die mans ook die natuurlike omgang 
met die vrou en brand van begeerte vir 
mekaar.  Mans pleeg skandelikhede met
mans en bring oor hulleself die verdien-
de straf vir hulle perversiteit.  28.  En omdat
hulle dit van geen belang ag om God te
ken nie, gee Hy hulle oor aan hulle ver-
draaide opvattings, sodat hulle doen wat
onbetaamlik is.  29.  Hulle is een en al onge-
regtigheid, slegtheid, hebsug en ge-
meenheid; hulle is vol jaloesie, moord,
twis, bedrog en kwaadwilligheid.  Hulle
skinder en 30.  praat kwaad; hulle haat 
God, hulle is hooghartig, aanmatigend, 
verwaand; hulle is mens wat kwaad uit-
dink, ongehoorsaam aan hulle ouers;
31.  hulle is onverstandig, onbetroubaar,
liefdeloos, hardvogtig.  32.  Hulle is mense 
wat die verordening van God ken dat dié
wat sulke dinge doen, die dood verdien,
en tog doen hulle nie net self hierdie dinge
nie, maar hulle vind dit ook goed as ander
dit doen.

ROM 2

Die regverdige oordeel van God

1.  Daarom is daar vir jou geen veront-
skuldiging nie, vir jou, mens wat 'n 
ander veroordeel, wie jy ook al is.  Deur-
dat jy oor \'n ander \'n oordeel uitspreek, 
veroordeel jy jouself, want jy wat veroor-
deel, doen dieselfde dinge.  2.  Ons weet dat
God regverdig handel wanneer Hy mense
wat sulke dinge doen, veroordeel.  3.  Maar
jy, mens, wat ander veroordeel wat sulke 
dinge doen, en jy doen dit self, verbeel jy 
jou dat jy aan die oordeel van God sal 
ontkom?  4.  God is ryk in goedheid, ver-
draagsaamheid en geduld!  Sê dit vir jou 
niks nie?  Besef jy nie dat God jou deur sy 
goedheid tot bekering wil lei nie?  5.  Maar
deur jou verharding en jou onbekeerlike 
hart is jy besig om vir jouself straf op te 
gaar vir die oordeelsdag, wanneer God 
sy regverdige oordeel sal uitspreek.  6.  Hy 
sal elkeen vergeld volgens sy dade:  7.  aan
dié wat in goedoen volhard en op dié 
manier soek na wat verhewe, eervol en 
onsterflik is, gee Hy die ewige lewe;
8.  maar dié wat uit selfsug ongehoorsaam
is aan die waarheid en toegee aan die on-
geregtigheid, straf Hy in sy toorn.  9.  Ly-
ding en benoudheid bring Hy oor die 
lewe van elke mens wat kwaad doen, in 
die eerste plek oor dié van die Jood, maar
ook oor dié van die nie-Jood.  10.  Wat 
verhewe en eervol is, en vrede skenk Hy 
aan elkeen wat goed doen, in die eerste
plek aan die Jood, maar ook aan die nie-
Jood.  11.  God trek immers niemand voor
nie.

Die besit van die wet stel jou nie vry van
straf nie

12.  Almal wat sonder die wet van Moses
gesondig het, sal ook sonder die wet ver-
lore gaan;  en almal wat onder die wet
gesondig het, oor hulle sal volgens die
wet geoordeel word.  13.  Nie dié wat die
wet hoor, word deur God vrygespreek
nie, maar dié wat doen wat die wet sê, sal
vrygespreek word.  14.  Wanneer heidene,
wat nie die wet het nie, tog vanself dinge
doen wat die wet vereis, is hulle vir hulle-
self 'n wet al het hulle nie die wet nie.
15.  Die optrede van sulke mense bewys dat
die eise van die wet in hulle harte geskry-
we staan.  Ook hulle gewetens getuig 
daarvan wanneer hulle in 'n innerlike 
tweestryd deur hulle gedagtes aange-
kla of vrygespreek word.  16.  Dit sal aan 
die lig kom op die dag wanneer God 
deur Christus Jesus oor die verbor-
ge dinge van die mense sal oordeel, oor-
eenkomstig die evangelie wat ek verkon-
dig.

Die Jode en die wet

17.  Jy, jy sê jy is 'n Jood, jy verlaat jou op
die wet van Moses, jy beroem jou daarop
dat God jou God is, 18.  jy ken sy wil en jy 
onderskei die dinge waar dit werklik op
aankom, omdat jy deur die wet onderrig
word.  19.  Jy is daarvan oortuig dat jy 'n 
gids vir die blindes is, 'n lig vir dié wat in
die duisternis is, 20.  'n leermeester van die 
wat onkundig is, 'n onderwyser van dié
wat niks weet nie, omdat die wet vir jou 
die samevatting is van kennis en waar-
heid - 21.  jy dan, wat 'n ander leer, leer jy 
jouself?  Jy wat preek dat 'n mens nie mag
steel nie, steel jy nie?  22.  Jy wat sê 'n mens 
mag nie egbreuk pleeg nie, pleeg jy nie 
egbreuk nie?  Jy wat 'n afsku het van af-
gode, beroof jy nie hulle tempels nie?
23.  Jy wat jou op die wet beroem, doen jy 
God nie oneer aan deur sy wet te oortree 
nie?  24.  Inderdaad, soos daar geskrywe 
staan, "as gevolg van julle optrede word
die Naam van God deur die heidennasies
belaster."

Die besnydenis sonder meer het nie
waarde nie

25.  Die besnydenis het waarde alleen as 
jy die wet van Moses onderhou.  Maar as
jy die wet voortdurend oortree, is jou 
besnydenis niks anders as onbesneden-
heid nie.  26.  As die onbesnedene die voor-
skrifte van die wet onderhou, sal God
dan nie sy onbesnedenheid as besneden-
heid reken nie?  27.  Ja, iemand wat krag-
tens sy afkoms 'n onbesnedene is maar 
tog die wet onderhou, sal jou selfs ver-
oordeel, vir jou wat met wetvoorskif-
te en besnydenis en al die wet oortree.
28.  Nie hy is 'n Jood wat dit uiter-
lik aan die liggaam gedoen is nie.  29.  Nee, 
hy is 'n Jood wat dit innerlik is, en 
dit is besnedenheid wat in die hart 
gedoen is deur die Gees, nie volgens
die wetsvoorskrifte nie.  So iemand ont-
vang lof, nie van mense nie, maar van 
God.


ROM 3



Die oordeel van God is nie in stryd met die
voorreg van die Jode nie.  

1.  Het dit dan enige voordeel om 'n 

Jood te wees, en het die besnydenis 

enige nut?  2.  Ja, baie, in allerlei opsigte.  

Die belangrikste is dat God sy woorde 

aan die Jode toevertrou het.  3.  Maar wat 

nou as sommige nie getrou gebly het nie?  

Sal hulle ontrou die trou van God laat 

ophou?  4.  Beslis nie.  Dit staan vas dat God 

betroubaar is en elke mens 'n leuenaar, 

soos daar geskrywe staan:  

"U is regverdig wanneer U 

uitspraak doen 

en U wen u saak wanneer U 

aangekla word."

5.  Ons verkeerde dade laat dus blyk dat 

God regverdig is.  Wat sal ons daarvan 

sê?  Ek redeneer soos mense gewoonlik 

redeneer.  Is God dan nie onregverdig as 

Hy ons straf nie?  6.  Beslis nie.  Hoe sou 

God anders oor die wêreld kan oordeel?  

7.  Maar as my leuenagtigheid die betrou-

baarheid van God duideliker aan die lig 

gebring het en so dien tot eer van God, 

waarom word ek dan nog as sondaar ver-

oordeel?  8.  Beteken dit dat ons sê:  Laat 

ons die verkeerde doen sodat die goeie 

daaruit kan voortkom?  Daar is mense 

wat ons dit wel lasterlik in die mond lê, 

maar hulle kry die oordeel wat hulle 

verdien.


Niemand is regverdig nie

9.  Waarop kom dit neer?  Is ons as Jode 
dan beter as die ander?  Glad nie, want 
ons het al bewys dat Jode en nie-Jode 
almal in die mag van die sonde is.  10.  Daar 
staan immers geskrywe:
"Daar is nie een wat regverdig is nie,
selfs nie een nie,
11.  daar is nie een wat verstandig is nie;
daar is nie een 
wat na die wil van God
vra nie.

12.  "Almal het afgedwaal,
almal het ontaard.
Daar is nie een wat goed doen nie,
selfs nie een nie.

13.  "Hulle keel is 'n oop graf, 
met hulle tonge bedrieg hulle.
Oor hulle lippe kom woorde 
so giftig soos slange, 
14.  hulle mond is vol vervloeking 
en bitterheid.

15.  "Hulle voete is vinnig om bloed 
te vergiet.
16.  Hulle laat 'n spoor van verwoesting 
en ellende agter.  
17.  Die pad van vrede het hulle 
nie leer ken nie, 
18.  ontsag vir God het hulle nie.

19.  Dit weet ons:  alles wat Moses se wet sê,
sê hy vir dié wat die wet het.  Niemand sal 
hom dus kan verweer nie, en die hele 
wêreld is strafwaardig voor God.  20.  Daar-
om sal geen mens op grond van wetsond-
derhouding deur God vrygespreek word 
nie;  inteendeel, deur die wet leer 'n mens
wat sonde is.

God verlos die mens

21.  Maar nou het die vryspraak deur 
God waarvan die wet en die profete ge-
tuig, in werking getree.  Dit is die vry-
spraak wat nie verkry word deur die wet 
te onderhou nie, 22.  maar deur in Jesus 
Christus te glo.  God gee dit sonder on-
derskeid aan almal wat glo.  23.  Almal het 
gesondig en is ver van God af.  24.  maar
hulle word, sonder dat hulle dit verdien, 
op grond van sy genade vrygespreek 
vanweë die verlossing deur Jesus Chris-
tus.  25-26.  Hom het God gegee as offer wat 
deur sy bloed versoening bewerk het 
vir dié wat glo.  Hierdeur het God ge-
toon wat sy vryspraak behels:  Hy het
die sondes wat Hy vooreen in sy ver-
draagsaamheid tydelik ongestraf laat 
bly het, vergewe.  Maar Hy het ook ge-
toon wat sy vryspraak in die teenswoor-
dige tyd behels:  Hy oordeel regverdig 
deurdat Hy elkeen vryspreek wat in Jesus 
glo.
27.  Het ons nou iets uit onsself om op te 
roem?  Nee, dit is uitgesluit.  Deur watter 
wet?  Dié van die werke?  Nee deur dié 
van die geloof.  28.  Ons betoog is tog dat 'n 
mens vrygespreek word omdat hy glo, 
nie omdat hy die wet onderhou nie.  29.  Of 
is God net God van die Jode, nie ook van 
die heidennasies nie?  Ja, ook van die hei-
dennasies, 30.  want daar is net een God.  
Hy sal die nesnedenes deur die geloof en 
die onbesnedenes deur dieselfde geloof 
vryspreek.  31.  Hef ons dan deur die geloof
die wet op?  Beslis nie.  Ons laat die wet 
juis tot sy reg kom.


ROM 4

Die voorbeeld van Abraham

1.  Wat moet ons nou sê was die geval 
met ons stamvader Abraham?  2.  As 
Abraham op grond van sy dade vryge-
spreek is, dan het hy 'n rede gehad om te 
roem.  Maar nie by God nie!  3.  Wat sê die 
Skrif?  "Abraham het in God geglo, en 
daarom het God hom vrygespreek."  4.  'n 
Arbeider se loon word nie vir hom as 'n 
guns gegee nie, maar as iets wat hom 
toekom.  5.  Maar die mens wat nie op wets-
onderhouding staatmaak nie, maar wat 
glo in Hom wat die goddelose vryspreek, 
hy kry die vryspraak deur sy geloof.  6.  So 
sê ook Dawid dat dié mens geseënd is wat 
deur God vrygespreek word buite wets-
onderhouding om:
7.  "Geseënd is die mense 
wie se oortredinge 
nie gestraf word nie 
en wie se sonde vergewe word.
8.  Geseënd is die mens vir wie die Here
die sonde nie toereken nie."
9.  Het hierdie uitspraak oor wie geseënd 
is, net betrekking op die besnedenes of 
ook op die onbesnedenes?  Ons sê weer:  
"Omdat Abraham geglo het, het God 
hom vrygespreek.  10.  Wanneer het dit ge-
beur?  Toe hy reeds besny was of toe hy 
nog onbesnede was?  Dit wa nie toe hy 
besnede was nie, maar toe hy nog onbe-
snede was.  11.  Hy het die besnydenis as 'n 
teken ontvang.  Dit is 'n seël wat bewys 
dat God hom vrygespreek het omdat hy 
geglo het toe hy nog onbesnede was.  Die 
doel daarmee was dat hy die vader sou 
wees van almal wat glo al is hulle nie 
besny nie.  Hulle word dus ook deur God 
vrygespreek omdat hulle glo.  12.  Ons 
voorvader Abraham sou ook die vader 
wees van dié besnedenes wat nie net be-
sny is nie maar ook glo, soos hy geglo het
toe hy nog onbesnede was.














ROM 13

Verpligtinge teenoor owerhede

1.  Elke mens moet hom onderwerp 
aan die owerhede wat oor hom 
gestel is.  Daar is immers geen gesag wat 
nie van God af kom nie, en die owerhede 
wat daar is, is daar deur die beskikking 
van God.  2.  Wie hom teen gesag verset, 
Kom dus in opstand teen die ordening 
van God; en wie in opstand kom, sal sy 
verdiende straf kry.  3.  'n Mens hoef nie vir 
die owerhede bang te wees as jy goed 
doen nie, maar wel as jy kwaad doen.  WIl 
jy sonder vrees vir die owerheid lewe?  
Doen dan wat goed is, en die owerheid 
sal jou prys, 4.  Want die owerheid is 'n 
dienaar van God tot jou beswil.  Maar as 
jy kwaad doen, het jy rede om bang te 
wees, want die owerheid het nie verniet 
die reg om te straf nie.  Hy is immers ook 
hierin die dienaar van God dat hy die 
kwaaddoener moet straf.  5.  Daarom moet 
jy jou onderwerp, nie net omdat jy vir 
straf bang is nie, maar ook omdat dit 'n 
gewetensaak is.
6.  Dit is ook waarom julle belasting be-
taal, want die owerhede is dienaars van 
God en hulle is besig om hulle opdrag uit 
te voer.  7.  Gee dus aan almal wat hulle 
toekom:  belasting as dit belasting is, ak-
syns as dit aksyns is, ontsag as dit ontsag 
is, eer as dit eer is.




1 KORINTIëRS

2 KORINTIëRS

GALASIëRS

EFESIëRS

FILIPPENSE

KOLOSSENSE

1 TESSALONISENSE

2 TESSALONISENSE

1 TIMOTEUS

2 TIMOTEUS

TITUS

FILEMON

HEBREëRS

JAKOBUS

1 PETRUS

2 PETRUS

1 JOHANNES

2 JOHANNES

3 JOHANNES

JUDAS

DIE OPENBARING VAN JOHANNES



\end{document}